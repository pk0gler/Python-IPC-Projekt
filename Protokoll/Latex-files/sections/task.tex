%!TEX root=../document.tex

\section{Einführung}
Diese Übung zeigt die Anwendung bzw. die Umsetzung eines Loadbalancers in Java.
Es sollen somit 2 verschiedene Loadbalancingverfahren umgesetzt werden.

\subsection{Ziele}
Das Ziel dieser Übung ist einen funktionierenden Loadbalancer in Java umzusetzten.

Dieser soll 2 verschiedene Loadbalancing Methoden implementieren.

Die Kommunikation zwischen Client und Server wird mithilfe von RMI umgesetzt.
Es soll möglich sein eine gewisse Anzahl von Clients und entsprechenden Server zu simulieren.
\subsection{Voraussetzungen}
\begin{itemize}
	\item Grundlagen Java
	\item Grundlagen in Loadbalancing
	\item Grundlagen mit Remote Method Invocation (RMI)
\end{itemize}

\subsection{Aufgabenstellung}
Es soll ein Load Balancer mit mindestens 2 unterschiedlichen Load-Balancing Methoden (jeweils 7 Punkte) implementiert werden (ähnlich dem PI Beispiel [1]; Lösung zum Teil veraltet [2]). Eine Kombination von mehreren Methoden ist möglich. Die Berechnung/Service und die Kommunikationstechnologie (Java RMI, CORBA, HTTP, Sockets, ...) sind frei wählbar!

Folgende Load Balancing Methoden stehen zur Auswahl:

\begin{itemize}
	\item Round Robin
	\item Weighted Round Robin
	\item Least Connection
	\item Agent Based / Server Probes
\end{itemize}

Um die Komplexität zu steigern, soll zusätzlich eine "Session Persistence" (2 Punkte) implementiert werden.
\clearpage
