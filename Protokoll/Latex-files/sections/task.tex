%!TEX root=../document.tex

\section{Einführung}
Erstelle einen Java- ODER Python-Client, der sich mit dem Server verbindet und selbstständige Entscheidungen trifft!

Dabei soll dieser Client dementsprechend protokolliert und der verwendete bzw. entworfene Algorithmus beschrieben werden.

Zusätzlich soll der Code des Servers analysiert werden und entsprechende Erweiterungen eingebaut werden.

\subsection{Ziele}
Das Ziel der Übung ist, Python und Sockets genauer kennen zu lernen.
Somit soll die Inter Prozess Kommunikation nahe gebracht werden.

Zusätzlich dient die Übung als EInblick bzw. Einstieg in das Programmieren einer einfachen KI welche selbst ständig agieren soll
\subsection{Voraussetzungen}
\begin{itemize}
	\item Grundlagen Python
	\item Grundlagen IPC
	\item Grundlagen Pathfinding
\end{itemize}

\subsection{Aufgabenstellung}
Es soll ein Client, welcher selbsständig Befehle an den Server schickt entworfen werden. Dabei soll darauf geachtet werden, dass dieser den Fallen bzw. den Lakes ausweicht.

DIeser Algorithmus ist entsprechend zu dokumentieren.

Zusätzlich sollen entsprechende Erweiterungen am Server durchgeführt werden.
\begin{itemize}
	\item Einstellen der Feldgroesse des Servers
	\item Einstellen der Felder (Mountain, Forest,Lakes)
\end{itemize}

\clearpage

Bei der Umsetzung des Clients sind folgende Punkte zu beachten.

\begin{itemize}
	\item Der Client schickt anschließend basierend auf den Antworten des Servers selbstständig (ohne Benutzereingaben) Move-Befehle:
	\begin{itemize}
		\item "UP": Nach oben bewegen
		\item "RIGHT": Nach rechts bewegen
		\item "DOWN": Nach unten bewegen
		\item "LEFT": Nach links bewegen
	\end{itemize}
	\item Der Client lässt die Spielfigur NICHT ins Wasser fallen
	\item Der Client bewegt sich zur Schriftrolle, wenn er sie sieht
	\item Der Client bewegt sich zur gegnerischen Burg, nachdem er die Schriftrolle eingesammelt hat
	\item Ansonsten erkundet der Client die Landschaft - auf der Suche nach Schriftrolle bzw. Burg
	\item Alle Verbindungen werden sauber geschlossen
	\item JavaDoc/DocString-Kommentare sowie Dokumentation (Sphinx/JavaDoc) sind vorhanden
\end{itemize}

\clearpage
