%!TEX root=../document.tex
\section{Allgemeine Beschreibung des Algorithmus}
\subsection{Erste Überlegnungen und Ziele}
Bei dem Entwurf des Algorithmus wurde darauf Wert gelegt den best möglichen Weg zu wählen.

Ursprünglich war die Überlegung, alle erhaltenen Felder zu speichern und nach jedem Zug neu zu bewerten. Bei der Bewertung werden sowohl die benötigten Züge (heuristic) zu diesem Node betrachtet, als auch die neuen Felder die dieser Node aufdecken würde.

Nach der Bewertung der einzelnen Nodes, erfolgt die Berechnung bzw. das Erstellen eines optimalen Weges zu diesem Node.
Um dies umzusetzen wurde der \textbf{A*} Algorithmus umgesetzt und dementsprechend implementiert. Die Beschreibung des eigentlichen Pathfinding wird im späteren Verlauf des Protokolls erklärt. 

Je nach Spielmodus, welcher von der KI definiert wird, berechnen sich neue günstige Felder.

\subsection{Definieren des aktuellen Spielmodus}
Die KI kann sich in 3 verschiedenen Spielmodi befinden.
Diese sind wie folgt:
\begin{itemize}
	\item \textbf{Aufdecken von Feldern | bevor Schriftrolle}\\
	Der erste Spielmodus beschreibt die Suche nach der Schriftrolle. Dabei werden möglichst viele Felder aufgedeckt um die Schriftrolle schnell zu finden.
	
	\item \textbf{Weg zur Schriftrolle}\\
	Nachdem die Schriftrolle gesichtet wurde, muss noch der optimale Weg zu dieser gefunden werden. Dabei muss beachtet werden, dass keine Lakes betreten werden, und dass das Feld keine Ränder hat und somit der direkte Weg nicht immer der schnellste ist.
	
	\item \textbf{Aufdecken von Feldern | nach Schriftrolle}\\
	Nachdem die Schriftrolle in Besitz ist, muss diese noch zu der gegnerischen Burg gebracht werden. Falls dieses noch nicht aufgedeckt wurde muss ein erneutes Suchen nach unbekannten Feldern durchgeführt werden.
	
	\item \textbf{Weg zur Burg}\\
	Sobald die gegnerische Burg gesichtet wurde, muss wieder der optimale Weg gefunden werden.
	Hierbei ist zu beachten, dass der optimale Weg hier durch die geringste Anzahl von Zügen beschrieben wird. Im Gegensatz zum optimalen Weg zur Schriftrolle wo sehr woll auch Felder aufgedeckt werden sollen um die gegnerische Burg eventuelle früher aufzudecken.
\end{itemize}

\subsection{Bewerten der Felder}
Bei der Bewertung eines Feldes werden mehrere Faktoren betrachtet.
Zusätzlich spielt auch der aktuelle Spielmodus eine Rolle.
Der wichtigste Faktor ist die möglichen Felder, welche diese Feld aufdecken würde. Zusätzlich wird der Weg bzw. die Anzahl zu diesem Feld in die Berechnung mit einbezogen.

\section{Implementierung des Algorithmus}

\subsection{Vorbereitung / Voraussetzung}
Um den Algorithmus best möglich zu Implementieren wurden gewisse Vorbereitungen gtroffen um dies zu erleichtern.

\subsubsection{Die Node Klasse}
\begin{minipage}{.5\textwidth}
\begin{center}
	\includegraphics[width=0.8\linewidth]{images/node.png}
\end{center}
\end{minipage}%
\begin{minipage}{.5\textwidth}
Die Node Klasse beschreibt einen Node also ein Feld in der erstellten Map.\\

Um das Bewerten und das Pathfinding zu erleichtern, hat die Node Klasse bestimmte Methoden.
Beispielsweise wird mit passable bekannt gegeben ob dieser Node begehbar ist oder nicht. Oder visited ob dieser schon besucht wurde.
\end{minipage}

\subsubsection{Die Priority Queue}
\begin{minipage}{.5\textwidth}
	Um bei dem Pathfinding Algorithmus den Nodes eine Gewichtung zu geben oder bei der Berrechnung der günstigen Felder in der Nähe eine Priorität zu vergeben, wurde die Priority Queue Klasse entworfen.
	Mithilfe der Library \textbf{heapq} können elemente mit \textbf{put} hinzugefügt und mit \textbf{get} das Element mit der höchsten Priorität entnommen werden
\end{minipage}
\begin{minipage}{.5\textwidth}
	\begin{center}
		\includegraphics[width=0.8\linewidth]{images/prio.png}
	\end{center}
\end{minipage}%


\subsection{Speichern der Map}
Bei dem Speichern der erhaltenen Felder,in einer Map dargestellt durch ein 2 Dimensionales Array, wird angenommen dass die eigene Burg gekennzeichnet mit C als Punkt mit den Koordinaten [0,0] behandelt wird.

Um nun alle neu erhaltenen Felder erneut einzuspeichern, muss beachtet werden, wie sich der Spieler im letzten Zug fortbewegt hat und die x bzw. y Position anpassen.

Zusätzlich muss beachtet werden welche Range also Sichtweite das aktuelle Feld liefert.

Die Speicherung wurde wie folgt umgesetzt:
\begin{lstlisting}[style=python, caption=Map Speichern]
# Anpassen des letzten Commands
# x und y demenstrpechend veraendern
else:
	if prev_command == "up":
		prev_y -= 1
	elif prev_command == "down":
		prev_y += 1
	elif prev_command == "left":
		prev_x -= 1
	elif prev_command == "right":
		prev_x += 1

prev_x = prev_x % size_x
prev_y = prev_y % size_y

# in der map speichern
# for schleife fuer y
for y in range(0, len(fields)):
	# for schleife fuer x
	for x in range(0, len(fields)):
		# temporaerer Node
		node_temp = fields[y][x]
		# Koordinaten anpassen
		node_temp.coor = ((x-len(fields)//2+prev_x) % size_x, (y-len(fields)//2+prev_y) % size_y)
		# wenn ein C enthalten ist und dieses nicht an 0, 0 steht
		# als Enemy Castle kennzeichnen
		if "C" in node_temp.value and node_temp.coor != (0,0):
			node_temp.value = "EC"
			print("enemy castle found")
		# Node in der Map eintragen
		map[(y - len(fields)//2+prev_y) % size_x][(x-len(fields)//2+prev_x) % size_y] = node_temp

# Kennzeichenen dass das Feld begangen wurde
map[prev_y][prev_x].visited = True
\end{lstlisting}

\subsubsection{Je nach Feld neue Gewichtung setzen}
Nach der erfolgreichen Speicherungen der erhaltenen Felder, müssen diese für den A* Algorithmus beurteilt werden.

Dafür wird die Gewichtung sowie ob dieser Node begehbar ist, in einem Graphen, welcher von dem A* Algorithmus verwendet wird, eingetragen
\begin{lstlisting}[style=python, caption=Graph fuer A* beurteilen]
# graph fuer a* neu beurteilen
for y in range(size_y):
	for x in range(size_x):
		node = map[y][x]
		if type(node) == Node:
			if not node.passable:
				# Falls der Node nicht begehbar ist also L
				graph.walls.append(node.coor)
		# Bewerten mit der entsprechenden Gewichtung
		graph.weights[node.coor] = node.weightv
\end{lstlisting}

\subsection{Bewerten der Felder}
Nach der berrechneten Gewichtung muss dass nächste zu besuchende Feld ermittelt werden.

Dafür werden alle Felder erneut bewertet und somit das optimale Feld ermittelt. In die Berrechnung wird der benötigte Weg zu diesem Node, als auch die möglichen aufzudeckenden Felder, mit einbezogen.

Die Unterscheidung findet durch den aktuellen Spielmodus statt

Wenn die bombe noch nicht gefunden wurde werden neue felder aufgedeckt also wird so bewertet dass das feld, welches die meisten neuen Felder aufdecken würde ganz oben in der Queue eingereiht wird.

Wenn die Bombe gefunden wurde aber das Gegnerische Castle noch nicht werden wieder Felder aufgedeckt. Falls nun die gegenerscihe Burg gefunden wurde, wird dieses mit der höchsten Priorität eingereiht.

\begin{minipage}{.5\textwidth}
	\hspace{0.9cm}\textbf{Bewerten der Felder nach Modus}
	\begin{lstlisting}[style=python, caption=Felder Bewerten]
 # map felder bewerten
 queue = PriorityQueue()
 # schelife fuer y
 for y in range(size_y):
	 # schleife fuer y
	 for x in range(size_x):
		 node = map[y][x]
		 if type(node) == Node:
			 # Nur Nodes die bekannt sind berechnen
			 # 0 wird ignoriert
			 if "B" in node.value and have_bomb == False:
				 # bombe gefunden prio ganz oben
				 print("bombe gefunden in die queue rein")
				 queue.put(node, -1000000000)
				 bomb_node = node.coor
			 elif "EC" in node.value and have_bomb == True:
				 # falss bombe schon aufgehoben 
				 # prio ganz oben
				 print("Gehe zu EC")
				 queue.put(node, -1000000000)
			 else:
				 # berechnen der aufdeckbaren felder
				 # durch das aktuelle feld
				 poss_fields = calc_possible_new_fields(node.value, node.coor, map)
			 way_to_field = heuristic(node.coor, (prev_x, prev_y))
			 queue.put(node, (poss_fields * -1) + way_to_field)
			 print(poss_fields, "for field:", node)
			 print("way to field:", way_to_field)
			 print()
 
 # Speichern des Feldes in der Queue
 next_node = queue.get()
 start = (prev_x, prev_y)
 goal = next_node.coor
	\end{lstlisting}
\end{minipage}%
\begin{minipage}{.5\textwidth}
	\hspace{0.7cm}\textbf{Berrechnen der aufdeckbaren Felder}
	\begin{lstlisting}[style=python2, caption=Mögliche Felder ermitteln]
def calc_possible_new_fields(val, coor, map):
	my_range = 0
	if "G" in val:
	my_range = 5
	elif "M" in val:
	my_range = 7
	elif "L" in val:
	my_range = 0
	else:
	my_range = 3

	fields_in_range = []

	for y in range(coor[1] - (my_range // 2), coor[1] + (my_range // 2 + 1)):
		for x in range(coor[0] - (my_range // 2), coor[0] + (my_range // 2 + 1)):
			coor_temp = (x % size_x, y % size_y)
			if coor_temp != coor:
				fields_in_range.append(coor_temp)
	
	# calculate new fields or not
	for i in fields_in_range:
		for y in range(size_y):
			for x in range(size_x):
				temp = map[y][x]
				if type(temp) == Node:
					if temp.coor == i:
						for ii in range(len(fields_in_range)):
							if fields_in_range[ii] == temp.coor:
								fields_in_range[ii] = 0
							else:
								pass
	
	fields_in_range = [x for x in fields_in_range if x != 0]
	
	return len(fields_in_range)
	\end{lstlisting}
\end{minipage}
